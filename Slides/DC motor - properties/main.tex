\documentclass{beamer}

\input{settings.tex}


\title{DC motor - properties}
\subtitle{Mechatronics, Lecture 5}
\author{by Sergei Savin}
\centering
\date{\mydate}



\begin{document}
\maketitle



\begin{frame}{Content}
\begin{itemize}
\item Steady State
\item Static State
\item Power
\item Efficiency
\item Static power
\item Reducer and power
\end{itemize}
\end{frame}





\begin{frame}{DC motor, steady state, 1}
	%\framesubtitle{How do we know the state?}
	\begin{flushleft}
		
		Dynamics of the motor with payload is:
		
		\begin{equation}
			\begin{cases}
				L \dot I + RI + C_w \omega = u \\
				J \dot \omega + F \omega = C_\tau I - \tau_p
			\end{cases}
		\end{equation}
	%
	where $\tau_p$ is the payload torque.
	
	\bigskip
	
	Let us consider stead-state $\omega = \omega^* = \text{const}$ and $I = \text{const}$:
	
	\begin{equation}
		\begin{cases}
			RI + C_w \omega^* = u \\
			F \omega^* = C_\tau I - \tau_p
		\end{cases}
	\end{equation}
	
		
	\end{flushleft}
\end{frame}



\begin{frame}{DC motor, steady state, 2}
	%\framesubtitle{How do we know the state?}
	\begin{flushleft}
		
		Knowing the desired angular velocity $ \omega^*$ and payload torque $\tau_p$ can solve the steady-state equations for current and voltage:
		
%		\begin{equation}
%			\begin{bmatrix}
%				R & -1 \\ C_\tau  & 0
%			\end{bmatrix}
%			\begin{bmatrix}
%				I \\ u
%			\end{bmatrix}
%		=
%		\begin{bmatrix}
%			-C_w \omega^* \\ F \omega^*-\tau_p
%		\end{bmatrix}
%		\end{equation}

\begin{align}
	I = \frac{F}{C_\tau} \omega^* + \frac{1}{C_\tau} \tau_p 
	\\
	u = \frac{FR + C_w C_\tau}{C_\tau} \omega^* + \frac{R}{C_\tau} \tau_p
\end{align}
		
		
	\end{flushleft}
\end{frame}



\begin{frame}{DC motor, static state}
	%\framesubtitle{How do we know the state?}
	\begin{flushleft}
		
		We can also consider a steady state equations with $\omega = 0$,  $\varphi = \text{const}$ and $I = \text{const}$. In this case, the current and voltage becomes:
		
		\begin{align}
			I = \frac{1}{C_\tau} \tau_p 
			\\
			u = \frac{R}{C_\tau} \tau_p
		\end{align}
		
		
	\end{flushleft}
\end{frame}




\begin{frame}{DC motor, power, 1}
	%\framesubtitle{How do we know the state?}
	\begin{flushleft}
		
		We can compute electrical power $P_e$ consumed by the DC motor:
		
		\begin{equation}
			P_e = I u
		\end{equation}
		
		The mechanical power $P_p$ associated with the payload:
		
		\begin{equation}
			P_p = \omega \tau_p
		\end{equation}
		
	\end{flushleft}
\end{frame}




\begin{frame}{DC motor, efficiency, 1}
	%\framesubtitle{How do we know the state?}
	\begin{flushleft}
		
		Multiplying steady-state equations by $I$ and $\omega$ we get:
		
		\begin{equation}
			\begin{cases}
				RI^2 + C_w I\omega = Iu \\
				F \omega^2 = C_\tau I\omega - \tau_p\omega
			\end{cases}
		\end{equation}
		
		Since $P_e = I u$ and $P_p = \omega \tau_p$ we get:
		
		\begin{equation}
			\begin{cases}
				P_e = RI^2 + C_w I\omega\\
				P_p = C_\tau I\omega - F \omega^2
			\end{cases}
		\end{equation}
		
		This allows us to compute \emph{efficiency} $P_p / P_e$:
		
		\begin{equation}
			\frac{P_p}{P_e} = \frac{C_\tau I\omega - F \omega^2}{C_w I\omega + RI^2}
		\end{equation}
	
		
		
	\end{flushleft}
\end{frame}



\begin{frame}{DC motor, efficiency, 2}
	%\framesubtitle{How do we know the state?}
	\begin{flushleft}
		
		Let us note that efficiency $\frac{P_p}{P_e}$ approaching 1 would mean that the entire power input is transformed to mechanical power output.
		
		\bigskip
		
		We note that $W_h = RI^2$ is the power of heating generated by the resistor (motor winding).  If we assume $F$ is small enough for $F \omega^2$ to be negligible ($F \omega^2 << |C_\tau I\omega|$), we get a simpler expression:
		
		\begin{equation}
			\frac{P_p}{P_e} = \frac{C_\tau I\omega}{C_w I\omega + W_h}
		\end{equation}
		
		It intuitively understandable that power of heating directly impedes the transformation of electrical power into mechanical.
		
		
	\end{flushleft}
\end{frame}



\begin{frame}{DC motor, efficiency, 3}
	%\framesubtitle{How do we know the state?}
	\begin{flushleft}
		
		 If we assume $W_h$ is small enough ($W_h << |C_w I\omega|$), we get a simpler expression:
		
		\begin{equation}
			\frac{P_p}{P_e} = \frac{C_\tau I\omega}{C_w I\omega} = \frac{C_\tau}{C_w}
		\end{equation}
		
		There is a common assumption that torque constant  is equal to the back-EMF constant: $C_\tau \sim C_w$. This gives us $\frac{P_p}{P_e} \sim 1$. 
		
		\bigskip
		
		We can view the ratio $\frac{C_\tau}{C_w}$ as an approximate efficiency measure, excluding the effects of viscous friction and heating power.
		
		
	\end{flushleft}
\end{frame}



\begin{frame}{DC motor, static power}
	%\framesubtitle{How do we know the state?}
	\begin{flushleft}
		
		Let us consider the power equations once more:
		
		\begin{equation}
			\begin{cases}
				P_e = RI^2 + C_w I\omega\\
				P_p = C_\tau I\omega - F \omega^2
			\end{cases}
		\end{equation}
		
		Assuming that the shaft remains still ($\omega = 0$) we get:
		
		\begin{equation}
			\begin{cases}
			P_e = RI^2 = W_h\\
			P_p = 0
			\end{cases}
		\end{equation}
		
		This means that while mechanical power is zero, electrical power is not. This equation illustrates that the entire power input in this regime is being converted into heat.
		
		
	\end{flushleft}
\end{frame}



\begin{frame}{Reducer effect}
	%\framesubtitle{How do we know the state?}
	\begin{flushleft}
		
		Let us note that reducer does not influence electrical power. We can show that ideal reducer does not affect the mechanical power either.
		
		\bigskip
		
		Let the output angular velocity be $\omega_p = \frac{1}{N}\omega$ and the output torque be $\tau_p = N\tau$. Then mechanical power becomes:
		
		\begin{equation}
			P_p = \frac{1}{N}\omega N\tau = \omega \tau
		\end{equation}
		
		Meaning that ideal reducer has no influence on mechanical power. Actual physical reducers do influence output power, making it lower. The information on the efficiency of a reducer is usually available. Note that the power loss in the reducer can be modeled by taking into account viscous friction associated with the gearbox.
		
	\end{flushleft}
\end{frame}


\begin{frame}{Read more}
	% \framesubtitle{Local coordinates}
	\begin{flushleft}
		
		\begin{itemize}
			\item \bref{https://www.researchgate.net/publication/281125444\_Modeling\_and\_design\_of\_geared\_DC\_motors\_for\_energy\_efficiency\_Comparison\_between\_theory\_and\_experiments}{Verstraten, T., Mathijssen, G., Furnemont, R., Vanderborght, B. and Lefeber, D., 2015. Modeling and design of geared DC motors for energy efficiency: Comparison between theory and experiments. Mechatronics, 30, pp.198-213.}
			
		\end{itemize}
		
		
	\end{flushleft}
\end{frame}


\myqrframe

\end{document}
