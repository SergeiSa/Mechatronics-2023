\documentclass{beamer}

\input{settings.tex}


\title{Second order systems}
\subtitle{Mechatronics, Lecture 6}
\author{by Sergei Savin}
\centering
\date{\mydate}



\begin{document}
\maketitle



\begin{frame}{Content}
\begin{itemize}
\item Eigenvalues
\item Characteristic Polynomial
\item Poles
\item Natural Frequency
\item Frequency response
\end{itemize}
\end{frame}





\begin{frame}{Second-order systems}
	% \framesubtitle{O}
	\begin{flushleft}
		
		Some of the second-order dynamical systems we have seen before:
		
		\begin{itemize}
			\item Spring-mass-damper dynamics.
			\item RLC circuit.
			\item DC motor angular velocity dynamics.
			\item Linearized model of a pendulum.
		\end{itemize}
	
		In a few lectures we will see that PID controller also acts in a way similar to a second-order dynamical system.
		
		\bigskip
		
		Since any polynomial can be represented as a product of linear polynomials and quadratic polynomials, we can also re-write any transfer function as a product of transfer functions with quadratic polynomials in the denominator. We can see second-order dynamical systems as fundamental building blocks of dynamical systems.
		
	\end{flushleft}
\end{frame}



\begin{frame}{Eigenvalues, 1}
	% \framesubtitle{O}
	\begin{flushleft}
		
		Observing eq. $m \ddot y + \mu_0 \dot y + c_0 y = 0$ we can tell that it is stable if (sufficient but not necessary condition) $m > 0$, $\mu > 0$, and $c > 0$ - this follows from the physics of the system.
		
		\bigskip
		
		A more principled approach is to find eigenvalues of the linear system. We start by dividing the equation by $m$:
		
		\begin{equation}
			\ddot y + \mu \dot y + c y = 0
		\end{equation}
		
		where $\mu = \mu_0 / m$ and $c = c_0 / m$. Defining $x_1 = y$ and $x_2 = \dot y$, the system can be equivalently represented as:
		
		\begin{equation}
			\begin{bmatrix}
				\dot x_1 \\ \dot x_2
			\end{bmatrix}
			=
			\begin{bmatrix}
				0 & 1 \\
				-c & -\mu
			\end{bmatrix} 
			\begin{bmatrix}
				x_1 \\ x_2
			\end{bmatrix}   
		\end{equation}
		
		
	\end{flushleft}
\end{frame}



\begin{frame}{Eigenvalues, 2}
	% \framesubtitle{O}
	\begin{flushleft}
		
		With linear system 
		$
		\begin{bmatrix}
			\dot x_1 \\ \dot x_2
		\end{bmatrix}
		=
		\begin{bmatrix}
			0 & 1 \\
			-c & -\mu
		\end{bmatrix} 
		\begin{bmatrix}
			x_1 \\ x_2
		\end{bmatrix}   
		$, we need to find its eigenvalues. We can use a formula for eigenvalues based on trace and determinant:
		
		\begin{equation}
			\lambda = \frac{T \pm \sqrt{T^2 - 4D} }{2}
		\end{equation}
		
		where $T$ is trace and $D$ is the determinant.
		
		\bigskip
		
		In our case $T = -\mu$ and $D = c$, and eigenvalues are:
		
		\begin{equation}
			\lambda = \frac{-\mu \pm \sqrt{\mu^2 - 4c} }{2}
		\end{equation}
		
		
	\end{flushleft}
\end{frame}



\begin{frame}{Eigenvalues, 3}
	% \framesubtitle{O}
	\begin{flushleft}
		
		Let us analyze eigenvalues $\lambda = \frac{-\mu \pm \sqrt{\mu^2 - 4c} }{2}$. We can see that if $\mu \geq 0$ and $c \geq 0$, there are only two scenarios: 
		
		\begin{enumerate}
			\item $\mu^2 - 4c \geq 0$, in which case $\sqrt{\mu^2 - 4c} \leq \mu$, the eigenvalues are purely real and negative.
			\item $\mu^2 - 4c < 0$, in which case $\sqrt{\mu^2 - 4c}$ is a purely imaginary number, the eigenvalues are complex with negative real parts.
		\end{enumerate}
		
		If $\mu \geq 0$ and $c = 0$, $\lambda_1 = -\mu$, $\lambda_2 = 0$, hence the system is marginally stable.
		
	\end{flushleft}
\end{frame}



\begin{frame}{Eigenvalues, 4}
	% \framesubtitle{O}
	\begin{flushleft}
		
		
		If $\mu \geq 0$ and $c < 0$, then $\sqrt{\mu^2 - 4c} \geq \mu$, and eigenvalues are purely real and one of them is positive, the system is unstable. If $\mu < 0$ and $c < 0$ at least one of the eigenvalues is still positive.
		
		\bigskip
		
		If $\mu < 0$ and $c \geq 0$, then again there are only two scenarios: 
		
		\begin{enumerate}
			\item $\mu^2 - 4c \geq 0$, in which case $\sqrt{\mu^2 - 4c} \leq \mu$, the eigenvalues are purely real and positive.
			\item $\mu^2 - 4c < 0$, in which case $\sqrt{\mu^2 - 4c}$ is a purely imaginary number, the eigenvalues are complex with positive real parts.
		\end{enumerate}
		
		\begin{definition}
			If $\mu \geq 0$ and $c \geq 0$ the system is stable, if $\mu < 0$ or $c < 0$ it is unstable.
		\end{definition}
		
	\end{flushleft}
\end{frame}




\begin{frame}{Characteristic polynomial}
	% \framesubtitle{O}
	\begin{flushleft}
		
		Going back to the eq. $\ddot y + \mu \dot y + c y = 0$ we can write characteristic eq. for it:
		
		\begin{equation}
			k^2 y + \mu k + c = 0
		\end{equation}
		
		Its roots are given by the formula:
		
		\begin{equation}
			k = \frac{-\mu \pm \sqrt{\mu^2 - 4c} }{2}
		\end{equation}
		
		As we can see, it is exactly the same as the determinant-trace formula.
		
		
	\end{flushleft}
\end{frame}



\begin{frame}{Transfer function poles}
	% \framesubtitle{O}
	\begin{flushleft}
		
		Now, let us consider the transfer function:
		
		\begin{equation}
			W(s) = \frac{1}{s^2 + \mu s + c }
		\end{equation}
		
		Poles of the transfer functions are solutions of the polynomial in its denominator; notice that this polynomial is exactly the same as the previously discussed characteristic polynomial. Meaning the poles of the transfer function are eigenvalues of the linear system.
		
		
	\end{flushleft}
\end{frame}



\begin{frame}{Natural frequency representation, 1}
	% \framesubtitle{O}
	\begin{flushleft}
		
		Another popular form of writing a second-order ODE is the following:
		
		\begin{equation}
			\ddot y + 2 \zeta \omega_n \dot y + \omega_n^2 y = 0
		\end{equation}
		%
		where $\omega_n$ is a \emph{natural frequency} and $\zeta$ is \emph{damping factor}. The quantity $\zeta \omega_n$ is called \emph{damping attenuation}. The relation between $\zeta, \omega_n$ and $\mu, c$ is:
		
		\begin{align}
			\omega_n = \sqrt{c},  \ \ \ \ \zeta = \frac{\mu}{2\sqrt{c}}
		\end{align}
	
		With this, the expression for eigenvalues is:
		
		\begin{equation}
			\lambda = -\zeta \omega_n \pm \omega_n \sqrt{\zeta^2 - 1}
		\end{equation}
		
	\end{flushleft}
\end{frame}


\begin{frame}{Natural frequency representation, 2}
	% \framesubtitle{O}
	\begin{flushleft}
		
		With the representation $\ddot y + 2 \zeta \omega_n \dot y + \omega_n^2 y = 0$ we can tell:
		
		\bigskip
		
		\begin{itemize}
			\item the system is overdamped (has pure real negative eigenvalues) is $\zeta > 1$, critically damped if $\zeta = 1$ and underdamped (produces oscillations) if  $0 \leq \zeta<1$;
			
			\item the system has purely imaginary eigenvalues (non-decaying oscillations) iff $\zeta = 0$ (since $\omega_n = 0$ corresponds to the trivial solution).
		\end{itemize}
		
		 
	\end{flushleft}
\end{frame}




\begin{frame}{Natural frequency representation, 3}
	% \framesubtitle{O}
	\begin{flushleft}
		
		\begin{figure}
			\centering
			\includegraphics[width=0.8\linewidth]{Poles}
			\caption{Poles and $\zeta, \omega_n$. \bref{https://web.mit.edu/2.14/www/Handouts/PoleZero.pdf}{Credit: MIT}.}
		\end{figure}
		
	\end{flushleft}
\end{frame}



\begin{frame}{Pure imaginary poles, 1}
	% \framesubtitle{O}
	\begin{flushleft}
		
		Given $\zeta = 0$ (implying $\mu = 0$) we have pure imaginary poles / eigenvalues $\lambda = \pm i \omega_n$. Moreover, the dynamics equations become:
		
		\begin{equation}
			\ddot y + \omega_n^2 y = 0
		\end{equation}
		
		Proposing solution $y = A \sin (\omega_n t + \varphi)$, noting that $\dot y = \omega_n A \cos (\omega_n t + \varphi)$ and $\ddot y = -\omega_n^2 A\sin (\omega_n t + \varphi)$, we see:
		
		\begin{align}
			-\omega_n^2 A\sin (\omega_n t + \varphi) + \omega_n^2 A \sin (\omega_n t + \varphi) = 0
		\end{align}
		%
		which is an equality.
		
		
	\end{flushleft}
\end{frame}




\begin{frame}{Pure imaginary poles, 2}
	% \framesubtitle{O}
	\begin{flushleft}
		
		Previous slide showed that the system without damping $\ddot y + \omega_n^2 y = 0$ will oscillate with angular frequency $\omega_n$; this motivates the name "natural frequency".
		
		\bigskip
		
		Also it shows that, if we pass a harmonic signal with angular frequency $\omega_n$ through a transfer function $W(s) = s^2 + \omega_n^2$ we will get a zero output.
		
		\bigskip
		
		Alternately, passing a harmonic signal with angular frequency $\omega_n$ through a transfer function $W(s) = \frac{1}{s^2 + \omega_n^2}$ we observe an infinite gain. Same can be observed by substituting $i \omega$ for $s$ and evaluating the absolute value of the transfer function at $\omega = \omega_n$. This effect is called \emph{resonance}.
		
		
	\end{flushleft}
\end{frame}




\begin{frame}{Frequency response}
	% \framesubtitle{O}
	\begin{flushleft}
		
		Given a transfer function $W(s) = \frac{1}{s^2 + 2 \zeta \omega_n s + \omega_n^2}$ we can find frequency response gain by computing absolute value of $W(i\omega)$:
		
		\begin{equation}
			G = |W(i\omega)| = \frac{1}{\sqrt{(\omega_n^2 - \omega^2)^2 + (2 \zeta \omega_n \omega)^2}}
		\end{equation}
	
	\begin{itemize}
		\item As $\omega \rightarrow 0$, $G \rightarrow \frac{1}{\omega_n^2}$.
		\item As $\omega \rightarrow \omega_n$, $G \rightarrow \frac{1}{\sqrt{2\zeta}\omega_n^2}$.
		\item As $\omega \rightarrow \infty$, $G \rightarrow 0$.
	\end{itemize}
		
		
	\end{flushleft}
\end{frame}



\begin{frame}{Read more}
	% \framesubtitle{Local coordinates}
	\begin{flushleft}
		
		\begin{itemize}
			\item \bref{https://web.mit.edu/2.14/www/Handouts/PoleZero.pdf}{MIT. 2.14 Analysis and Design of Feedback Control Systems. Understanding Poles and Zeros}
			
		\end{itemize}
		
		
	\end{flushleft}
\end{frame}


\myqrframe

\end{document}
