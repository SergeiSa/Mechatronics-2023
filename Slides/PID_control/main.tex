\documentclass{beamer}

\input{settings.tex}


\title{PID Control}
\subtitle{Mechatronics, Lecture 7}
\author{by Sergei Savin}
\centering
\date{\mydate}



\begin{document}
\maketitle



\begin{frame}{Content}
\begin{itemize}
\item PID Control
\item DC motor model
\item PD Control of a DC motor velocity
\item PD Control with reference signal
\item PID Control of a DC motor velocity
\item DC motor model with additive disturbance
\item PID Control of a DC motor velocity
\item PID Control, Laplace
\item PID Control - position
\end{itemize}
\end{frame}




\begin{frame}{PID Control}
	% \framesubtitle{O}
	\begin{flushleft}
		
		One of the most standard and widely used control laws is proportional-integral-derivative (PID) control. It is especially suitable for:
		
		\begin{itemize}
			\item SISO control, decentralized control, low-level control loop
			\item stabilizing control,
			\item shaping frequency response,
			\item shaping performance / step response,
			\item robust control (rejecting constant additive disturbances).
		\end{itemize}
		
		Proportional-derivative (PD) control is quite similar (sans robustness), and is widely used in theoretical research.
		
	\end{flushleft}
\end{frame}



\begin{frame}{DC motor model}
	% \framesubtitle{O}
	\begin{flushleft}
		
		Dynamics of the DC motor can be represented as:
		%
		\begin{align}
			JL \ddot \omega  + (LF+JR) \dot \omega   +( FR+C_w C_\tau)\omega =
			C_\tau u 
		\end{align}
		
		We can re-write the model in new variables:
		%
		\begin{align}
			\ddot \omega  + a\dot \omega  + c\omega =
			b u 
		\end{align}
		%
		where $a = \frac{LF+JR}{JL}$, $c = \frac{FR+C_w C_\tau}{JL}$, and $b = \frac{C_\tau}{JL}$.
		
	\end{flushleft}
\end{frame}


\begin{frame}{PD Control of a DC motor velocity}
	% \framesubtitle{O}
	\begin{flushleft}
		
		If we want to control angular velocity with a \emph{PD control}, the control law will take form:
		
		\begin{equation}
			u = - K_p \omega - K_d \dot \omega
		\end{equation}
	
		Substituting the control law into the dynamics equations gives us \emph{closed-loop dynamics}:
		%
		\begin{align}
			\ddot \omega  + a\dot \omega  + c\omega =
			- b K_p \omega - b K_d \dot \omega 
		\end{align}
		
		Grouping the terms we get:
		%
		\begin{align}
			\ddot \omega  + (a + b K_d)\dot \omega  + (c + b K_p)\omega = 0
		\end{align}
		
		We can manipulate coefficients $K_p$ and $K_d$ to achieve desired behavior of the system.
		
	\end{flushleft}
\end{frame}



\begin{frame}{PD Control example}
	% \framesubtitle{O}
	\begin{flushleft}
		
		Consider the following dynamical system:
		
		\begin{align}
			\ddot \omega  + 2 \dot \omega  + 5\omega = 0.5 u
		\end{align}
		
		We will attempt to find PD control law that turns it into a system $\ddot \omega  + 5 \dot \omega  + 10\omega = 0$.
		
		\bigskip
		
		We need to solve the following linear equations:
		%
		\begin{align}
			2 + 0.5 K_d = 5 \\
			5 + 0.5 K_p = 10
		\end{align}
		
		giving us $K_d = 6$ and $K_p = 10$ and PD control law:
		%
		\begin{align}
			u = - 10 \omega - 6 \dot \omega
		\end{align}
		
	\end{flushleft}
\end{frame}



\begin{frame}{PD Control with reference signal, 1}
	% \framesubtitle{O}
	\begin{flushleft}
		
		Often we use control to follow a \emph{reference signal} $\omega_r(t)$. Control law in that case takes form:
		
		\begin{equation}
			u = K_p (\omega_r(t) - \omega) + K_d (\dot \omega_r(t) - \dot \omega)
		\end{equation}
		
		Substituting it into dynamics equation $\ddot \omega  + a\dot \omega  + c\omega =
		b u $ we get:
		
		\begin{align}
			\ddot \omega  + a\dot \omega  + c\omega =
			b K_p (\omega_r(t) - \omega) + b K_d (\dot \omega_r(t) - \dot \omega)
			\\
			\ddot \omega  + (a + b K_d )\dot \omega  + (c + b K_p)\omega =
			b K_p \omega_r(t) + b K_d \dot \omega_r(t)
		\end{align}
		
		
	\end{flushleft}
\end{frame}


\begin{frame}{PD Control with reference signal, 2}
	% \framesubtitle{O}
	\begin{flushleft}
		
		We can transform the last equation into Laplace domain:
		
		\begin{align}
			(s^2  + (a + b K_d ) s  + (c + b K_p))\omega(s) =
			(b K_p + b K_d s) \omega_r(s)
		\end{align}
	
		We find transfer function from the reference signal to the angular velocity $\omega(s)$:
		
		\begin{align}
			W_r(s) = \frac{b K_d s + b K_p}{s^2  + (a + b K_d ) s  + (c + b K_p)} \\
			\omega(s) = W_r(s) \omega_r(s)
		\end{align}
		
	\end{flushleft}
\end{frame}



\begin{frame}{DC motor model with additive disturbance}
	% \framesubtitle{O}
	\begin{flushleft}
		
		Sometimes it is hard to model the motor exactly; In particular, this can be expressed by considering additive disturbance:
		%
		\begin{align}
			JL \ddot \omega  + (LF+JR) \dot \omega   +( FR+C_w C_\tau)\omega =
			C_\tau u + d
		\end{align}
		
		where $d$ is the additive disturbance. We can re-write the model in new variables:
		%
		\begin{align}
			\ddot \omega  + a\dot \omega  + c\omega =
			b u + d.
		\end{align}
		
	\end{flushleft}
\end{frame}


\begin{frame}{PID Control of a DC motor velocity, 1}
	% \framesubtitle{O}
	\begin{flushleft}
		
		If we want to control angular velocity with a \emph{PID control}, the control law will take form:
		
		\begin{equation}
			u(t) =  - K_d \dot \omega(t) - K_p \omega(t) - K_i \int_0^t \omega(\tau) d\tau
		\end{equation}
		
		Defining $\varphi$ such that $\dot \varphi = \omega$ we get:
		
		\begin{equation}
			u(t) = - K_d \ddot \varphi(t) - K_p \dot \varphi(t) - K_i \varphi(t)
		\end{equation}
		
		Substituting the control law into the dynamics equations gives us \emph{closed-loop dynamics}:
		%
		\begin{align}
			\dddot \varphi  + a\ddot \varphi  + c\dot\varphi =
			d - bK_d \ddot \varphi - bK_p \dot \varphi - bK_i \varphi
		\end{align}
		
		Grouping the terms we get:
		%
		\begin{align}
			\dddot \varphi  + (a+bK_d)\ddot \varphi  + (c+bK_p)\dot\varphi + bK_i \varphi = d
		\end{align}
		
	\end{flushleft}
\end{frame}



\begin{frame}{PID Control of a DC motor velocity, 2}
	% \framesubtitle{O}
	\begin{flushleft}
		
		Considering the steady state of the equation $\dddot \varphi  + (a+bK_d)\ddot \varphi  + (c+bK_p)\dot\varphi + bK_i \varphi = d$, we get:
		%
		\begin{align}
			bK_i \varphi = d
		\end{align}
	
		With that, we can find steady-state value of $\varphi = d / (bK_i)$. Notice that it allows steady state solution with $\omega = 0$; the steady state value of $\varphi$ is irrelevant for the angular velocity control. This is the idea behind the integral part of PID control.
		
	\end{flushleft}
\end{frame}




\begin{frame}{PID Control, Laplace}
	% \framesubtitle{O}
	\begin{flushleft}
		
		PID control in Laplace domain looks like:
		
		\begin{equation}
			u(s) =  - (K_d s + K_p + \frac{K_i}{s}) \omega(s)
		\end{equation}
		
		With reference signal, PID control leads to the transfer function (from reference to angular velocity):
		
		\begin{align}
			W_r(s) = b\frac{K_d s^2 + K_ps + K_i}{s^3  + (a+bK_d ) s^2  + (c+bK_p)s + bK_i}
		\end{align}
		
	\end{flushleft}
\end{frame}




\begin{frame}{PID Control - position}
	% \framesubtitle{O}
	\begin{flushleft}
		
	If we want to control orientation of motor shaft, we have to re-write the dynamics in terms of $\varphi$. Since $\varphi(s) = \frac{1}{s} \omega(s)$, and $\omega(s) = \frac{b}{s^2 + a s+ c} u(s)$, we get:
	%
	\begin{equation}
		\varphi(s) = \frac{b}{s^3 + a s^2 + cs} u(s)
	\end{equation}
		
	The PID control will take the usual form:
	%
	\begin{equation}
		u(s) =  (K_d s + K_p + \frac{K_i}{s}) (\varphi_r(s) - \varphi(s))
	\end{equation}

	The closed-loop transfer function will be:
	%
	\begin{equation}
		\varphi(s) = b\frac{K_d s^2 + K_ps + K_i}{s^4 + a s^3 + (c+bK_d)s^2 + bK_p s + bK_i} \varphi_r(s)
	\end{equation}
	
		
	\end{flushleft}
\end{frame}


\myqrframe

\end{document}
