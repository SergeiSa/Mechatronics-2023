\documentclass{beamer}

\pdfmapfile{+sansmathaccent.map}


\mode<presentation>
{
	\usetheme{Warsaw} % or try Darmstadt, Madrid, Warsaw, Rochester, CambridgeUS, ...
	\usecolortheme{seahorse} % or try seahorse, beaver, crane, wolverine, ...
	\usefonttheme{serif}  % or try serif, structurebold, ...
	\setbeamertemplate{navigation symbols}{}
	\setbeamertemplate{caption}[numbered]
} 


%%%%%%%%%%%%%%%%%%%%%%%%%%%%
% itemize settings


%%%%%%%%%%%%%%%%%%%%%%%%%%%%
% itemize settings

\definecolor{myhotpink}{RGB}{255, 80, 200}
\definecolor{mywarmpink}{RGB}{255, 60, 160}
\definecolor{mylightpink}{RGB}{255, 80, 200}
\definecolor{mypink}{RGB}{255, 30, 80}
\definecolor{mydarkpink}{RGB}{155, 25, 60}

\definecolor{mypaleblue}{RGB}{240, 240, 255}
\definecolor{mylightblue}{RGB}{120, 150, 255}
\definecolor{myblue}{RGB}{90, 90, 255}
\definecolor{mygblue}{RGB}{70, 110, 240}
\definecolor{mydarkblue}{RGB}{0, 0, 180}
\definecolor{myblackblue}{RGB}{40, 40, 120}

\definecolor{myblackturquoise}{RGB}{5, 53, 60}
\definecolor{mydarkdarkturquoise}{RGB}{8, 93, 110}
\definecolor{mydarkturquoise}{RGB}{28, 143, 150}
\definecolor{mypaleturquoise}{RGB}{230, 255, 255}
\definecolor{myturquoise}{RGB}{48, 213, 200}

\definecolor{mygreen}{RGB}{0, 200, 0}
\definecolor{mydarkgreen}{RGB}{0, 120, 0}
\definecolor{mygreen2}{RGB}{245, 255, 230}

\definecolor{mygrey}{RGB}{120, 120, 120}
\definecolor{mypalegrey}{RGB}{160, 160, 160}
\definecolor{mydarkgrey}{RGB}{80, 80, 160}

\definecolor{mydarkred}{RGB}{160, 30, 30}
\definecolor{mylightred}{RGB}{255, 150, 150}
\definecolor{myred}{RGB}{200, 110, 110}
\definecolor{myblackred}{RGB}{120, 40, 40}

\definecolor{mygreen}{RGB}{0, 200, 0}
\definecolor{mygreen2}{RGB}{205, 255, 200}

\definecolor{mydarkcolor}{RGB}{60, 25, 155}
\definecolor{mylightcolor}{RGB}{130, 180, 250}

\setbeamertemplate{itemize items}[default]

\setbeamertemplate{itemize item}{\color{myblackturquoise}$\blacksquare$}
\setbeamertemplate{itemize subitem}{\color{mydarkdarkturquoise}$\blacktriangleright$}
\setbeamertemplate{itemize subsubitem}{\color{mygray}$\blacksquare$}

\setbeamercolor{palette quaternary}{fg=white,bg=myblackturquoise}
\setbeamercolor{titlelike}{parent=palette quaternary}

\setbeamercolor{palette quaternary2}{fg=black,bg=mypaleblue}
\setbeamercolor{frametitle}{parent=palette quaternary2}

\setbeamerfont{frametitle}{size=\Large,series=\scshape}
\setbeamerfont{framesubtitle}{size=\normalsize,series=\upshape}





%%%%%%%%%%%%%%%%%%%%%%%%%%%%
% block settings

\setbeamercolor{block title}{bg=red!30,fg=black}

\setbeamercolor*{block title example}{bg=mygreen!40!white,fg=black}

\setbeamercolor*{block body example}{fg= black, bg= mygreen2}


%%%%%%%%%%%%%%%%%%%%%%%%%%%%
% URL settings
\hypersetup{
	colorlinks=true,
	linkcolor=blue,
	filecolor=blue,      
	urlcolor=blue,
}

%%%%%%%%%%%%%%%%%%%%%%%%%%

\renewcommand{\familydefault}{\rmdefault}

\usepackage{amsmath}
\usepackage{mathtools}

\usepackage{subcaption}

\usepackage{qrcode}

\DeclareMathOperator*{\argmin}{arg\,min}
\newcommand{\bo}[1] {\mathbf{#1}}

\newcommand{\R}{\mathbb{R}} 
\newcommand{\T}{^\top}     



\newcommand{\mydate}{Fall 2023}

\newcommand{\mygit}{\textcolor{blue}{\href{https://github.com/SergeiSa/Control-Theory-Slides-Spring-2023}{github.com/SergeiSa/Control-Theory-Slides-Spring-2023}}}

\newcommand{\myqr}{ \textcolor{black}{\qrcode[height=1.5in]{https://github.com/SergeiSa/Control-Theory-Slides-Spring-2023}}
}

\newcommand{\myqrframe}{
	\begin{frame}
		\centerline{Lecture slides are available via Github, links are on Moodle}
		\bigskip
		\centerline{You can help improve these slides at:}
		\centerline{\mygit}
		\bigskip
		\myqr
	\end{frame}
}


\newcommand{\bref}[2] {\textcolor{blue}{\href{#1}{#2}}}

%%%%%%%%%%%%%%%%%%%%%%%%%%%%
% code settings

\usepackage{listings}
\usepackage{color}
% \definecolor{mygreen}{rgb}{0,0.6,0}
% \definecolor{mygray}{rgb}{0.5,0.5,0.5}
\definecolor{mymauve}{rgb}{0.58,0,0.82}
\lstset{ 
	backgroundcolor=\color{white},   % choose the background color; you must add \usepackage{color} or \usepackage{xcolor}; should come as last argument
	basicstyle=\footnotesize,        % the size of the fonts that are used for the code
	breakatwhitespace=false,         % sets if automatic breaks should only happen at whitespace
	breaklines=true,                 % sets automatic line breaking
	captionpos=b,                    % sets the caption-position to bottom
	commentstyle=\color{mygreen},    % comment style
	deletekeywords={...},            % if you want to delete keywords from the given language
	escapeinside={\%*}{*)},          % if you want to add LaTeX within your code
	extendedchars=true,              % lets you use non-ASCII characters; for 8-bits encodings only, does not work with UTF-8
	firstnumber=0000,                % start line enumeration with line 0000
	frame=single,	                   % adds a frame around the code
	keepspaces=true,                 % keeps spaces in text, useful for keeping indentation of code (possibly needs columns=flexible)
	keywordstyle=\color{blue},       % keyword style
	language=Octave,                 % the language of the code
	morekeywords={*,...},            % if you want to add more keywords to the set
	numbers=left,                    % where to put the line-numbers; possible values are (none, left, right)
	numbersep=5pt,                   % how far the line-numbers are from the code
	numberstyle=\tiny\color{mygray}, % the style that is used for the line-numbers
	rulecolor=\color{black},         % if not set, the frame-color may be changed on line-breaks within not-black text (e.g. comments (green here))
	showspaces=false,                % show spaces everywhere adding particular underscores; it overrides 'showstringspaces'
	showstringspaces=false,          % underline spaces within strings only
	showtabs=false,                  % show tabs within strings adding particular underscores
	stepnumber=2,                    % the step between two line-numbers. If it's 1, each line will be numbered
	stringstyle=\color{mymauve},     % string literal style
	tabsize=2,	                   % sets default tabsize to 2 spaces
	title=\lstname                   % show the filename of files included with \lstinputlisting; also try caption instead of title
}


%%%%%%%%%%%%%%%%%%%%%%%%%%%%
% URL settings
\hypersetup{
	colorlinks=false,
	linkcolor=blue,
	filecolor=blue,      
	urlcolor=blue,
}

%%%%%%%%%%%%%%%%%%%%%%%%%%

%%%%%%%%%%%%%%%%%%%%%%%%%%%%
% tikz settings

\usepackage{tikz}
\tikzset{every picture/.style={line width=0.75pt}}


\title{Mechanics, Dynamics}
\subtitle{Mechatronics, Lecture 1}
\author{by Sergei Savin}
\centering
\date{\mydate}



\begin{document}
\maketitle



\begin{frame}{Content}
\begin{itemize}
\item Introduction
\item Newton - Euler laws
\item Moment of inertia
\item Mechanical Energy, Work, Power
\end{itemize}
\end{frame}


\begin{frame}{Introduction}
	%\framesubtitle{How do we know the state?}
	\begin{flushleft}
		
		
		What is Mechatronics? The term \emph{Mechatronics} is a combination of words \emph{mechanics} and \emph{electronics}.
		
		\bigskip
		
		\begin{definition}
			Mechatronics is a synergistic integration of mechanical and electrical engineering, computer control in design and manufacturing. 
		\end{definition}
		
		
	\end{flushleft}
\end{frame}


\begin{frame}{Introduction}
	%\framesubtitle{How do we know the state?}
	\begin{flushleft}
		
		
		Examples of mechatronic design include:
		
		\begin{itemize}
			\item Motors with built-in gears and sensors.
			
			\item Quadrotor brushless motors.
			
			\item Motor-wheel.
		\end{itemize}		
	
		\bigskip
		
		Modern robots, from robot-arm to Boston Dynamics's ATLAS are also examples of mechatronic design.
		
		
	\end{flushleft}
\end{frame}



\begin{frame}{Introduction}
	%\framesubtitle{How do we know the state?}
	\begin{flushleft}
		
		
		In this course we will focus on:
		
		\begin{itemize}
			\item Motors: working principles, control, sensing;
			
			\item Gears, reducers, transmissions: models, friction and other physical effects;
			
			\item Single input, single output (SISO) control, as applicable to simple electro-mechanical systems;
			
			\item Sensors, using sensor data.
			
		\end{itemize}		
		
		
	\end{flushleft}
\end{frame}



\begin{frame}{Newton - Euler law}
%\framesubtitle{How do we know the state?}
\begin{flushleft}

Second Newton's law in 2D case:

\begin{equation}
	\begin{cases}
		m \ddot x = \sum f_{i, x} \\
		m \ddot y = \sum f_{i, y}
	\end{cases}
\end{equation}
%
where $m$ is mass of a body, $f_{i, x}$, $f_{i, y}$ - Cartesian components of forces acting on the body. 

\end{flushleft}
\end{frame}



\begin{frame}{Newton - Euler law}
	%\framesubtitle{How do we know the state?}
	\begin{flushleft}
		
		Euler law of motion in 2D case:
		
		\begin{equation}
			J \ddot \varphi = \sum \tau_i
		\end{equation}
		%
		where $J$ is a moment of inertia of a body, $\varphi$ is orientation of the body, $\tau_i$ is torque. All of these entities are defined assuming rotation around the axis normal to the plane of motion, with positive direction of rotation defined as clock-wise.
		
		\bigskip
		
		We can see similarity of these equation with Newton laws. Note that this is only true for a 2D case, not for a general 3D case, where the Euler dynamics equations have the following form:
		
		\begin{equation}
			\bo{I} \dot{\omega} = \omega \times (\bo{I} \omega) + \tau
		\end{equation}
	%
	where $\bo{I}$ is matrix of inertia, $\omega$ is angular velocity and $\tau$ is external torque (all in body frame).
		
	\end{flushleft}
\end{frame}




\begin{frame}{Pendulum dynamics}
	%\framesubtitle{How do we know the state?}
	\begin{flushleft}
		
		A pendulum can be described by the following dynamical equations:
		
		\begin{equation}
			J \ddot \varphi = lmg \sin (\varphi)
		\end{equation}		
		%
		where $J$, $l$, $m$ and $g$ are moment of inertia, length, mass and gravitational acceleration. 
		
		\bigskip
		
		If there is motor torque $\tau$ acting on the pendulum, the dynamics takes form:
		
		\begin{equation}
			J \ddot \varphi = lmg \sin (\varphi) + \tau
		\end{equation}		
		
		
	\end{flushleft}
\end{frame}



\begin{frame}{Torque generated by a force}
	%\framesubtitle{How do we know the state?}
	\begin{flushleft}
		
		A force $\bo{f}$ applied to a rigid body at a point $\bo{r}$ generates a torque $\tau$:
		
		\begin{equation}
			\tau = \bo{r} \times \bo{f}
		\end{equation}	
		
		In 2D case, a force with magnitude $f$ acting along a line with a distance $r$ from the axis (around which torque is computed) generates torque according to the formula:
		
		\begin{equation}
			\tau = r f
		\end{equation}	
		
	\end{flushleft}
\end{frame}


\begin{frame}{Moment of inertia, 1}
%\framesubtitle{How do we know the state?}
\begin{flushleft}
	
	 It is tempting to think of moment of inertia in 2D case as an analog of mass; moment of inertia determines the rate of change of \textcolor{blue}{angular velocity} for a given \textcolor{blue}{torque}, same as mass determines the rate of change of \textcolor{mydarkturquoise}{linear velocity} for a given \textcolor{mydarkturquoise}{force}.
	 
	 \begin{equation*}
		\begin{cases}
			\textcolor{blue}{J \dot \omega =  \tau}  \\
			\textcolor{mydarkturquoise}{m \dot v = f}
	\end{cases}
\end{equation*}
	 
	
\end{flushleft}
\end{frame}




\begin{frame}{Moment of inertia, 2}
	%\framesubtitle{How do we know the state?}
	\begin{flushleft}
		
		Another way to think about moment of inertia is through kinetic energy. Kinetic energy of a rigid body can be described as:
		
		\begin{equation}
			T = \frac{1}{2} m \bo{v}\T \bo{v} + \frac{1}{2} \omega\T \bo{I} \omega
		\end{equation}			 
		
		In 2D case it is simplified:
		
		\begin{equation}
			T = \frac{1}{2} m (v_x^2 + v_y^2) + \frac{1}{2} J \omega^2
		\end{equation}		
		
		Moment of inertia determines how energy will the body acquire by the increase in its angular velocity.
		
	\end{flushleft}
\end{frame}



\begin{frame}{Moment of inertia, 3}
	%\framesubtitle{How do we know the state?}
	\begin{flushleft}
		
		Moment of inertia of a point mass $r$ meters away from the axis of rotation is given as:
		
		\begin{equation}
			J =m  r^2
		\end{equation}				
	
		The same formula give moment of inertia for a rim of a disk with radius $r$. For a solid disk with uniform mass distribution the formula becomes:
		
		\begin{equation}
			J =\frac{1}{2} m  r^2
		\end{equation}				
		
		For a centrally mounted rod the moment of inertia is $J =\frac{1}{12} m  r^2$ and for a rod mounted like a pendulum it is $J =\frac{1}{3} m  r^2$.
		
	\end{flushleft}
\end{frame}



\begin{frame}{Mechanical Energy}
	%\framesubtitle{How do we know the state?}
	\begin{flushleft}
		
		\emph{Mechanical energy} is a sum of kinetic and potential energy. We already saw the form of kinetic energy in 2D case: $T = \frac{1}{2} m (v_x^2 + v_y^2) + \frac{1}{2} J \omega^2$. 
		
		\bigskip
		
		Potential energy is more complex: it depends on what type of potential (conservative) forces that act on the system. For example:
		
		\begin{itemize}
			\item gravitational potential energy: $\Pi = mg (y - y_0)$, where $y_0$ is the value of vertical coordinate $y$ for which potential energy is defined as zero;
			
			\item linear spring potential energy: $\Pi = \frac{1}{2} k (r - r_0)^2$, where $k$ is the stiffness and $r_0$ is the length of the spring in the relaxed state.
		\end{itemize}
		
	\end{flushleft}
\end{frame}


\begin{frame}{Mechanical Work, 1}
	%\framesubtitle{How do we know the state?}
	\begin{flushleft}
		
		The change of the total energy of the system is equal to the sum of \emph{mechanical work} performed by the active forces:
		
		\begin{equation}
			\Delta T + \Delta \Pi = \sum A_i
		\end{equation}				
		%
		where $A_i$ is the work performed by the active force $f_i$ or a torque $\tau_i$.
		
		\bigskip
		
		We could compute the work performed by a force (applied to a point mass) on the interval of time from $t_0$ to $t_f$:
		
		\begin{equation}
			A = \int_{t_0}^{t_f} \bo{v}\T \bo{f} \  dt
		\end{equation}
		%
		where $\bo{v} = \bo{v}(t)$	is the velocity of the point mass and $\bo{f} = \bo{f}(t)$ is the force.
		
	\end{flushleft}
\end{frame}



\begin{frame}{Mechanical Work, 2}
	%\framesubtitle{How do we know the state?}
	\begin{flushleft}
		
		If a force $\bo{f}$ is collinear with velocity $\bo{v}$, we re-write the formula for mechanical work using the absolute values of these vectors $f$ and $v = \dot s$:
		%
		\begin{equation}
			A = \int_{t_0}^{t_f} v f \  dt
		\end{equation}
		
		As long as $f = \text{const}$, the expression becomes trivial:
		%
		\begin{equation}
			A =  (s(t_1) - s(t_0)) f
		\end{equation}
	
		We can derive the same formulas for mechanical work of a torque $\tau$:
		%
		\begin{align}
			A &= \int_{t_0}^{t_f} \omega \tau \  dt
			 \\
			A &= (\varphi(t_1) - \varphi(t_0)) \tau \ \ \ \textcolor{mygrey}{(\tau =  \text{const})} \
		\end{align}
	
	\end{flushleft}
\end{frame}


\begin{frame}{Mechanical Power}
	%\framesubtitle{How do we know the state?}
	\begin{flushleft}
		
		The entity $\bo{v}\T \bo{f}$ we saw before is called \emph{mechanical power}. It can also be defined as:
		
		\begin{align}
			P &= \bo{v}\T \bo{f}
			\\
			P &= \omega \tau
			\\
			P &= \frac{d}{dt} A
		\end{align}
		
		Note that where as mechanical work is an integral entity, mechanical power is instantaneous. Also, using the definition of mechanical energy we observe that:
		%
		\begin{equation}
			\frac{d}{dt} (T + \Pi) = \sum P_i
		\end{equation}	
		
	\end{flushleft}
\end{frame}


\myqrframe

\end{document}
