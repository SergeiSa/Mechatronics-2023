\documentclass{beamer}

\pdfmapfile{+sansmathaccent.map}


\mode<presentation>
{
	\usetheme{Warsaw} % or try Darmstadt, Madrid, Warsaw, Rochester, CambridgeUS, ...
	\usecolortheme{seahorse} % or try seahorse, beaver, crane, wolverine, ...
	\usefonttheme{serif}  % or try serif, structurebold, ...
	\setbeamertemplate{navigation symbols}{}
	\setbeamertemplate{caption}[numbered]
} 


%%%%%%%%%%%%%%%%%%%%%%%%%%%%
% itemize settings


%%%%%%%%%%%%%%%%%%%%%%%%%%%%
% itemize settings

\definecolor{myhotpink}{RGB}{255, 80, 200}
\definecolor{mywarmpink}{RGB}{255, 60, 160}
\definecolor{mylightpink}{RGB}{255, 80, 200}
\definecolor{mypink}{RGB}{255, 30, 80}
\definecolor{mydarkpink}{RGB}{155, 25, 60}

\definecolor{mypaleblue}{RGB}{240, 240, 255}
\definecolor{mylightblue}{RGB}{120, 150, 255}
\definecolor{myblue}{RGB}{90, 90, 255}
\definecolor{mygblue}{RGB}{70, 110, 240}
\definecolor{mydarkblue}{RGB}{0, 0, 180}
\definecolor{myblackblue}{RGB}{40, 40, 120}

\definecolor{myblackturquoise}{RGB}{5, 53, 60}
\definecolor{mydarkdarkturquoise}{RGB}{8, 93, 110}
\definecolor{mydarkturquoise}{RGB}{28, 143, 150}
\definecolor{mypaleturquoise}{RGB}{230, 255, 255}
\definecolor{myturquoise}{RGB}{48, 213, 200}

\definecolor{mygreen}{RGB}{0, 200, 0}
\definecolor{mydarkgreen}{RGB}{0, 120, 0}
\definecolor{mygreen2}{RGB}{245, 255, 230}

\definecolor{mygrey}{RGB}{120, 120, 120}
\definecolor{mypalegrey}{RGB}{160, 160, 160}
\definecolor{mydarkgrey}{RGB}{80, 80, 160}

\definecolor{mydarkred}{RGB}{160, 30, 30}
\definecolor{mylightred}{RGB}{255, 150, 150}
\definecolor{myred}{RGB}{200, 110, 110}
\definecolor{myblackred}{RGB}{120, 40, 40}

\definecolor{mygreen}{RGB}{0, 200, 0}
\definecolor{mygreen2}{RGB}{205, 255, 200}

\definecolor{mydarkcolor}{RGB}{60, 25, 155}
\definecolor{mylightcolor}{RGB}{130, 180, 250}

\setbeamertemplate{itemize items}[default]

\setbeamertemplate{itemize item}{\color{myblackturquoise}$\blacksquare$}
\setbeamertemplate{itemize subitem}{\color{mydarkdarkturquoise}$\blacktriangleright$}
\setbeamertemplate{itemize subsubitem}{\color{mygray}$\blacksquare$}

\setbeamercolor{palette quaternary}{fg=white,bg=myblackturquoise}
\setbeamercolor{titlelike}{parent=palette quaternary}

\setbeamercolor{palette quaternary2}{fg=black,bg=mypaleblue}
\setbeamercolor{frametitle}{parent=palette quaternary2}

\setbeamerfont{frametitle}{size=\Large,series=\scshape}
\setbeamerfont{framesubtitle}{size=\normalsize,series=\upshape}





%%%%%%%%%%%%%%%%%%%%%%%%%%%%
% block settings

\setbeamercolor{block title}{bg=red!30,fg=black}

\setbeamercolor*{block title example}{bg=mygreen!40!white,fg=black}

\setbeamercolor*{block body example}{fg= black, bg= mygreen2}


%%%%%%%%%%%%%%%%%%%%%%%%%%%%
% URL settings
\hypersetup{
	colorlinks=true,
	linkcolor=blue,
	filecolor=blue,      
	urlcolor=blue,
}

%%%%%%%%%%%%%%%%%%%%%%%%%%

\renewcommand{\familydefault}{\rmdefault}

\usepackage{amsmath}
\usepackage{mathtools}

\usepackage{subcaption}

\usepackage{qrcode}

\DeclareMathOperator*{\argmin}{arg\,min}
\newcommand{\bo}[1] {\mathbf{#1}}

\newcommand{\R}{\mathbb{R}} 
\newcommand{\T}{^\top}     



\newcommand{\mydate}{Fall 2023}

\newcommand{\mygit}{\textcolor{blue}{\href{https://github.com/SergeiSa/Control-Theory-Slides-Spring-2023}{github.com/SergeiSa/Control-Theory-Slides-Spring-2023}}}

\newcommand{\myqr}{ \textcolor{black}{\qrcode[height=1.5in]{https://github.com/SergeiSa/Control-Theory-Slides-Spring-2023}}
}

\newcommand{\myqrframe}{
	\begin{frame}
		\centerline{Lecture slides are available via Github, links are on Moodle}
		\bigskip
		\centerline{You can help improve these slides at:}
		\centerline{\mygit}
		\bigskip
		\myqr
	\end{frame}
}


\newcommand{\bref}[2] {\textcolor{blue}{\href{#1}{#2}}}

%%%%%%%%%%%%%%%%%%%%%%%%%%%%
% code settings

\usepackage{listings}
\usepackage{color}
% \definecolor{mygreen}{rgb}{0,0.6,0}
% \definecolor{mygray}{rgb}{0.5,0.5,0.5}
\definecolor{mymauve}{rgb}{0.58,0,0.82}
\lstset{ 
	backgroundcolor=\color{white},   % choose the background color; you must add \usepackage{color} or \usepackage{xcolor}; should come as last argument
	basicstyle=\footnotesize,        % the size of the fonts that are used for the code
	breakatwhitespace=false,         % sets if automatic breaks should only happen at whitespace
	breaklines=true,                 % sets automatic line breaking
	captionpos=b,                    % sets the caption-position to bottom
	commentstyle=\color{mygreen},    % comment style
	deletekeywords={...},            % if you want to delete keywords from the given language
	escapeinside={\%*}{*)},          % if you want to add LaTeX within your code
	extendedchars=true,              % lets you use non-ASCII characters; for 8-bits encodings only, does not work with UTF-8
	firstnumber=0000,                % start line enumeration with line 0000
	frame=single,	                   % adds a frame around the code
	keepspaces=true,                 % keeps spaces in text, useful for keeping indentation of code (possibly needs columns=flexible)
	keywordstyle=\color{blue},       % keyword style
	language=Octave,                 % the language of the code
	morekeywords={*,...},            % if you want to add more keywords to the set
	numbers=left,                    % where to put the line-numbers; possible values are (none, left, right)
	numbersep=5pt,                   % how far the line-numbers are from the code
	numberstyle=\tiny\color{mygray}, % the style that is used for the line-numbers
	rulecolor=\color{black},         % if not set, the frame-color may be changed on line-breaks within not-black text (e.g. comments (green here))
	showspaces=false,                % show spaces everywhere adding particular underscores; it overrides 'showstringspaces'
	showstringspaces=false,          % underline spaces within strings only
	showtabs=false,                  % show tabs within strings adding particular underscores
	stepnumber=2,                    % the step between two line-numbers. If it's 1, each line will be numbered
	stringstyle=\color{mymauve},     % string literal style
	tabsize=2,	                   % sets default tabsize to 2 spaces
	title=\lstname                   % show the filename of files included with \lstinputlisting; also try caption instead of title
}


%%%%%%%%%%%%%%%%%%%%%%%%%%%%
% URL settings
\hypersetup{
	colorlinks=false,
	linkcolor=blue,
	filecolor=blue,      
	urlcolor=blue,
}

%%%%%%%%%%%%%%%%%%%%%%%%%%

%%%%%%%%%%%%%%%%%%%%%%%%%%%%
% tikz settings

\usepackage{tikz}
\tikzset{every picture/.style={line width=0.75pt}}


\title{PID Control}
\subtitle{Mechatronics, Lecture 7}
\author{by Sergei Savin}
\centering
\date{\mydate}



\begin{document}
\maketitle



\begin{frame}{Content}
\begin{itemize}
\item Steady State
%\item Static State
%\item Power
%\item Efficiency
%\item Static power
%\item Reducer and power
\end{itemize}
\end{frame}




\begin{frame}{PID Control}
	% \framesubtitle{O}
	\begin{flushleft}
		
		One of the most standard and widely used control laws is proportional-integral-derivative (PID) control. It is especially suitable for:
		
		\begin{itemize}
			\item SISO control, decentralized control, low-level control loop
			\item stabilizing control,
			\item shaping frequency response,
			\item shaping performance / step response,
			\item robust control (rejecting constant additive disturbances).
		\end{itemize}
		
		Proportional-derivative (PD) control is quite similar (sans robustness), and is widely used in theoretical research.
		
	\end{flushleft}
\end{frame}



\begin{frame}{DC motor model}
	% \framesubtitle{O}
	\begin{flushleft}
		
		Dynamics of the DC motor can be represented as:
		%
		\begin{align}
			JL \ddot \omega  + (LF+JR) \dot \omega   +( FR+C_w C_\tau)\omega =
			C_\tau u 
		\end{align}
		
		We can re-write the model in new variables:
		%
		\begin{align}
			\ddot \omega  + a\dot \omega  + c\omega =
			b u 
		\end{align}
		%
		where $a = \frac{LF+JR}{JL}$, $c = \frac{FR+C_w C_\tau}{JL}$, and $b = \frac{C_\tau}{JL}$.
		
	\end{flushleft}
\end{frame}


\begin{frame}{PD Control of a DC motor velocity}
	% \framesubtitle{O}
	\begin{flushleft}
		
		If we want to control angular velocity with a \emph{PD control}, the control law will take form:
		
		\begin{equation}
			u = - K_p \omega - K_d \dot \omega
		\end{equation}
	
		Substituting the control law into the dynamics equations gives us \emph{closed-loop dynamics}:
		%
		\begin{align}
			\ddot \omega  + a\dot \omega  + c\omega =
			- b K_p \omega - b K_d \dot \omega 
		\end{align}
		
		Grouping the terms we get:
		%
		\begin{align}
			\ddot \omega  + (a + b K_d)\dot \omega  + (c + b K_p)\omega = 0
		\end{align}
		
		We can manipulate coefficients $K_p$ and $K_d$ to achieve desired behavior of the system.
		
	\end{flushleft}
\end{frame}



\begin{frame}{PD Control example}
	% \framesubtitle{O}
	\begin{flushleft}
		
		Consider the following dynamical system:
		
		\begin{align}
			\ddot \omega  + 2 \dot \omega  + 5\omega = 0.5 u
		\end{align}
		
		We will attempt to find PD control law that turns it into a system $\ddot \omega  + 5 \dot \omega  + 10\omega = 0$.
		
		\bigskip
		
		We need to solve the following linear equations:
		%
		\begin{align}
			2 + 0.5 K_d = 5 \\
			5 + 0.5 K_p = 10
		\end{align}
		
		giving us $K_d = 6$ and $K_p = 10$ and PD control law:
		%
		\begin{align}
			u = - 10 \omega - 6 \dot \omega
		\end{align}
		
	\end{flushleft}
\end{frame}



\begin{frame}{PD Control with reference signal, 1}
	% \framesubtitle{O}
	\begin{flushleft}
		
		Often we use control to follow a \emph{reference signal} $\omega_r(t)$. Control law in that case takes form:
		
		\begin{equation}
			u = K_p (\omega_r(t) - \omega) + K_d (\dot \omega_r(t) - \dot \omega)
		\end{equation}
		
		Substituting it into dynamics equation $\ddot \omega  + a\dot \omega  + c\omega =
		b u $ we get:
		
		\begin{align}
			\ddot \omega  + a\dot \omega  + c\omega =
			b K_p (\omega_r(t) - \omega) + b K_d (\dot \omega_r(t) - \dot \omega)
			\\
			\ddot \omega  + (a + b K_d )\dot \omega  + (c + b K_p)\omega =
			b K_p \omega_r(t) + b K_d \dot \omega_r(t)
		\end{align}
		
		
	\end{flushleft}
\end{frame}


\begin{frame}{PD Control with reference signal, 2}
	% \framesubtitle{O}
	\begin{flushleft}
		
		We can transform the last equation into Laplace domain:
		
		\begin{align}
			(s^2  + (a + b K_d ) s  + (c + b K_p))\omega(s) =
			(b K_p + b K_d s) \omega_r(s)
		\end{align}
	
		We find transfer function from the reference signal to the angular velocity $\omega(s)$:
		
		\begin{align}
			W_r(s) = \frac{b K_d s + b K_p}{s^2  + (a + b K_d ) s  + (c + b K_p)} \\
			\omega(s) = W_r(s) \omega_r(s)
		\end{align}
		
	\end{flushleft}
\end{frame}



\begin{frame}{DC motor model with additive disturbance}
	% \framesubtitle{O}
	\begin{flushleft}
		
		Sometimes it is hard to model the motor exactly; In particular, this can be expressed by considering additive disturbance:
		%
		\begin{align}
			JL \ddot \omega  + (LF+JR) \dot \omega   +( FR+C_w C_\tau)\omega =
			C_\tau u + d
		\end{align}
		
		where $d$ is the additive disturbance. We can re-write the model in new variables:
		%
		\begin{align}
			\ddot \omega  + a\dot \omega  + c\omega =
			b u + d.
		\end{align}
		
	\end{flushleft}
\end{frame}


\begin{frame}{PID Control of a DC motor velocity, 1}
	% \framesubtitle{O}
	\begin{flushleft}
		
		If we want to control angular velocity with a \emph{PID control}, the control law will take form:
		
		\begin{equation}
			u(t) =  - K_d \dot \omega(t) - K_p \omega(t) - K_i \int_0^t \omega(\tau) d\tau
		\end{equation}
		
		Defining $\varphi$ such that $\dot \varphi = \omega$ we get:
		
		\begin{equation}
			u(t) = - K_d \ddot \varphi(t) - K_p \dot \varphi(t) - K_i \varphi(t)
		\end{equation}
		
		Substituting the control law into the dynamics equations gives us \emph{closed-loop dynamics}:
		%
		\begin{align}
			\dddot \varphi  + a\ddot \varphi  + c\dot\varphi =
			d - bK_d \ddot \varphi - bK_p \dot \varphi - bK_i \varphi
		\end{align}
		
		Grouping the terms we get:
		%
		\begin{align}
			\dddot \varphi  + (a+bK_d)\ddot \varphi  + (c+bK_p)\dot\varphi + bK_i \varphi = d
		\end{align}
		
	\end{flushleft}
\end{frame}



\begin{frame}{PID Control of a DC motor velocity, 2}
	% \framesubtitle{O}
	\begin{flushleft}
		
		Considering the steady state of the equation $\dddot \varphi  + (a+bK_d)\ddot \varphi  + (c+bK_p)\dot\varphi + bK_i \varphi = d$, we get:
		%
		\begin{align}
			bK_i \varphi = d
		\end{align}
	
		With that, we can find steady-state value of $\varphi = d / (bK_i)$. Notice that it allows steady state solution with $\omega = 0$; the steady state value of $\varphi$ is irrelevant for the angular velocity control. This is the idea behind the integral part of PID control.
		
	\end{flushleft}
\end{frame}




\begin{frame}{PID Control, Laplace}
	% \framesubtitle{O}
	\begin{flushleft}
		
		PID control in Laplace domain looks like:
		
		\begin{equation}
			u(s) =  - (K_d s + K_p + \frac{K_i}{s}) \omega(s)
		\end{equation}
		
		With reference signal, PID control leads to the transfer function (from reference to angular velocity):
		
		\begin{align}
			W_r(s) = b\frac{K_d s^2 + K_ps + K_i}{s^3  + (a+bK_d ) s^2  + (c+bK_p)s + bK_i}
		\end{align}
		
	\end{flushleft}
\end{frame}




\begin{frame}{PID Control - position}
	% \framesubtitle{O}
	\begin{flushleft}
		
	If we want to control orientation of motor shaft, we have to re-write the dynamics in terms of $\varphi$:
	%
	\begin{equation}
		\varphi(s) = \frac{b}{s^3 + a s^2 + cs} u(s)
	\end{equation}
		
	The PID control will take the usual form:
	%
	\begin{equation}
		u(s) =  (K_d s + K_p + \frac{K_i}{s}) (\varphi_r(s) - \varphi(s))
	\end{equation}
	%
	\begin{equation}
		\varphi(s) = \frac{1}{s^3 + a s^2 + cs} u(s)
	\end{equation}

	The closed-loop transfer function will be:
	%
	\begin{equation}
		\varphi(s) = b\frac{K_d s^2 + K_ps + K_i}{s^4 + a s^3 + (c+bK_d)s^2 + bK_p s + bK_i} \varphi_r(s)
	\end{equation}
	
		
	\end{flushleft}
\end{frame}


\myqrframe

\end{document}
